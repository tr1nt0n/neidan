\documentclass[12pt]{article}
\usepackage{fontspec}
\usepackage[utf8]{inputenc}
\setmainfont{Bodoni 72 Book}
\usepackage[paperwidth=12in,paperheight=9in,margin=1in,headheight=0.0in,footskip=0.5in,includehead,includefoot,portrait]{geometry}
\usepackage[absolute]{textpos}
\TPGrid[0.5in, 0.25in]{23}{24}
\parindent=0pt
\parskip=12pt
\usepackage{nopageno}
\usepackage{graphicx}
\graphicspath{ {./images/} }
\usepackage{amsmath}
\usepackage{hyperref}
\usepackage{tikz}
\newcommand*\circled[1]{\tikz[baseline=(char.base)]{
            \node[shape=circle,draw,inner sep=1pt] (char) {#1};}}

\begin{document}

\begin{center}
\huge NOTES FOR THE INTERPRETERS
\end{center}

\begingroup
\textbf{\circled{1} Tablature} is used at various moments throughout the piece. The clefs of this tablature are as follows: \\
\circled{1.} \includegraphics[scale=0.55]{body_clef.png} \textbf{The body clef} indicates to touch the body of the instrument. The top line represents the top of the body, the bottom line the bottom. The center line represents the waist. \\ 
\circled{2.} \includegraphics[scale=0.45]{back_clef.png} \textbf{The back of body clef} indicates to play on the back of the body. This clef is always coupled with a technique wherein the bow is pressed firmly on the back of the instrument and twisted, causing the hairs to rub against the wood of the bow, the sound of which is amplified by the resonating body. \\ 
\circled{3.} \includegraphics[scale=0.35]{string_clef.png} \textbf{The string clef} is not a far cry from similar tablatures used by Helmut Lachenmann. The top line indicates the very top of the string, the second line represents halfway down the string, the third line represents the edge of the fingerboard, the fourth line represents the bridge, and the final line presents the tailpiece.
\endgroup 

\begingroup
\textbf{In the tablature idiom}, stringing is given in the form of roman numerals next to the note which behave exactly as accidentals. \\
\textbf{\circled{2} Note heads} correspond to finger pressure. The note heads used in this score, apart from the traditional, are: \\
\circled{1.} \includegraphics[scale=0.55]{harmonic.png} Harmonic finger pressure. \\
\circled{2.} \includegraphics[scale=0.55]{half_harmonic.png} Half-harmonic finger pressure ( effectively damping the string ) \\
\circled{3.} \includegraphics[scale=0.35]{cross.png} Percussive action, such as striking the instrument with the bow or fingertips ( often unpitched )
\endgroup

\begingroup
\textbf{The presence of two staves} indicates to divide the actions indicated in the notation into right hand ( top staff ) and left hand ( bottom staff ). 
\endgroup

\begingroup
\textbf{Not all actions of the tablature will sound.} In many cases, it is interpreted just as choreographically as it is sonically. 
\endgroup

\begingroup
\textbf{\circled{2}} In various passages throughout this piece, there is notation which represents \textbf{the point at which the bow is touched} as it is drawn across the string. These positions are written as \textbf{fractions} and \textbf{0/5} represents \textbf{au talon} and and \textbf{5/5} represents \textbf{punta d`arco}. For the duration of the note to which these fractions are attached, the interpreter should draw the bow at a constant speed, moving toward the destination point indicated on the following note. Bowings are provided. Passages without these indications should be bowed at the interpreter’s discretion. 
\endgroup

\begingroup
\textbf{\circled{3} Materials} Required for this score are: \\
\circled{1.} A small basin which can fit comfortably in the interpreters lap. \\
\circled{2.} At least five, preferably uncarved pieces of wood, which are placed in the basin. \\
\circled{3.} A pitcher of water. \\
\circled{4.} A metal guitar slide.
\endgroup

\begingroup
\textbf{\circled{4}} The third movement is intended to be played with the metal guitar slide. The indication of finger position and pressure in the left hand staff can either be maintained, accomplished by careful angling and rotating of the slide, or interpreted with one finger operating the slide and the other fingers touching the strings.  
\endgroup

\begingroup
\textbf{\circled{5}} This piece should be performed sitting down, with the instrument in da gamba position, as at the ending of the choreography which begins the first movement. 
\endgroup

\begingroup
\textbf{\circled{6} The instrument is prepared} with small rings of aluminum foil loosely wrapped around each of the strings beneath the bridge. The fourth string is detuned a quarter tone, to C-quarter-flat.  
\endgroup

\pagebreak

\textbf{\circled{7}} This score follows the notational example of Luigi Nono, who used the familiar, round-arched \textbf{fermata} as an orientation point. To \textbf{triangulate} the arch indicates to \textbf{shorten} the fermata, and to \textbf{square} the arch indicates to \textbf{lengthen} the fermata. The \textbf{addition of arcs} increases the relative length or shortness of the fermata. Sometimes fermate are given dashed arches, indicating to slightly shorten or elongate the fermata, but not fully to the level of the proceeding symbol. \\
The interpreters are advised against quantization via the development of their own system for counting fermatas ( in seconds, for example ). Instead, the fermata should be understood as an invitation to wait rather than count, the shape of their arcs being an indication of the relative space of this invitation.

\end{document}